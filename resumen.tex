\chapter*{Resumen}
\addcontentsline{toc}{chapter}{Resumen}

El ojo es una fuente de información valiosa que, al ser analizada rigurosamente, permite no solo obtener resultados biométricos únicos, sino también aplicar estos hallazgos en áreas diversas. Bajo esta premisa, esta investigación desarrolló un sistema de seguimiento ocular de bajo costo capaz de identificar personas y controlar interfaces mediante la mirada.

La metodología empleó un enfoque híbrido que combina características físicas con patrones dinámicos de movimiento (velocidad y precisión). Las pruebas con 15 participantes arrojaron una exactitud de identificación del 83.5\%, superando a los métodos que solo utilizan imágenes estáticas. Como aplicación práctica, se validó la viabilidad técnica del sistema para el control de cursor en tecnologías asistivas.

Adicionalmente, el análisis reveló que los patrones oculares varían según el nivel de cansancio e iluminación, sugiriendo un nuevo uso potencial para el monitoreo de fatiga en salud ocupacional. En conclusión, este trabajo demuestra que es factible implementar soluciones avanzadas de biometría y accesibilidad sin depender de hardware costoso.

\vspace{0.5cm}

\textbf{Palabras clave:} Seguimiento ocular, biometría, procesamiento de imágenes, tecnologías asistivas, inteligencia artificial.