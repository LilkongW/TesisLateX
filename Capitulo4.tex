% =============================================================================
% CAPÍTULO IV
% =============================================================================
\chapter{Resultados y Discusión}

% --- INTRODUCCIÓN DEL CAPÍTULO ---
En este capítulo se presentan los resultados obtenidos tras el procesamiento y análisis de las señales oculomotoras de los 15 participantes del estudio. La exposición se organiza en cinco etapas fundamentales: primero, se valida la calidad técnica de la señal capturada y el rendimiento del algoritmo de detección; segundo, se caracteriza la dinámica fisiológica de los movimientos registrados; tercero, se evalúa la capacidad discriminativa de las métricas biométricas propuestas; cuarto, se presenta el rendimiento de los modelos de clasificación automática; y finalmente, se analiza la viabilidad del sistema para aplicaciones de control de cursor.

\section{Validación del Sistema de Captura y Procesamiento}

Antes de abordar el análisis biométrico, es fundamental verificar la integridad de los datos adquiridos y la robustez del sistema de visión artificial implementado.

\subsection{Calidad de la Señal y Filtrado}

El análisis inicial de los datos crudos reveló la presencia de ruido de
alta frecuencia, un problema común en sensores CMOS cuando operan con
ganancia variable en el espectro infrarrojo. Holmqvist et al.\
\cite{holmqvist2011eye} señalan que este tipo de \textit{jitter}
instrumental es inherente a los sistemas de captura basados en cámaras
no especializadas, y que su mitigación mediante filtrado digital es un
paso indispensable antes de cualquier análisis cinemático, dado que las
derivadas temporales (velocidad, aceleración) amplifican exponencialmente
el ruido de alta frecuencia si se calculan sobre la señal cruda.

Para mitigar este \textit{jitter} sin comprometer la integridad de la
información biológica, se aplicó el filtro digital Savitzky-Golay.
Savitzky y Golay \cite{savitzky1964smoothing} demostraron que el ajuste
polinomial local sobre ventanas deslizantes preserva los momentos
estadísticos de la señal —media, varianza y curtosis— con mayor fidelidad
que los filtros de promedio móvil convencionales, precisamente porque el
polinomio se adapta a la curvatura local de la señal en lugar de
promediarla hacia cero. La configuración óptima del filtro se estableció
con una ventana de longitud $w=21$ muestras y un polinomio de orden
$p=3$. Esta elección de parámetros es crítica para el estudio:

\begin{itemize}
	\item \textbf{Ventana ($w=21$)}: A una tasa de 120 FPS, esta ventana
	abarca un contexto temporal de $\approx 175$ ms. Según el criterio de
	Nyquist \cite{nyquist1928certain}, la frecuencia de muestreo de 120 Hz
	permite reconstruir fielmente componentes de señal de hasta 60 Hz, rango
	que cubre toda la dinámica fisiológica relevante del sistema oculomotor.
	La ventana de 175 ms proporciona suavizado suficiente para eliminar
	fluctuaciones aleatorias del centroide pupilar sin suprimir los eventos
	sacádicos, cuya duración mínima supera los 20 ms \cite{bahill1975}.
	
	\item \textbf{Polinomio Cúbico ($p=3$)}: A diferencia de los filtros de
	promedio móvil que tienden a atenuar o recortar los picos de señal, el
	ajuste polinomial de tercer grado preserva los momentos de inercia y,
	crucialmente, la magnitud real de la velocidad durante los movimientos
	sacádicos rápidos. Harwood, Mezey y Harris \cite{harwood1999spectral}
	confirmaron que el contenido frecuencial de los sacádicos se concentra
	por debajo de los 50 Hz, lo que hace al polinomio cúbico una
	aproximación suficientemente flexible para capturar esa dinámica sin
	sobreajustar el ruido.
\end{itemize}

Como se evidencia en la Figura~\ref{fig:filtrado_signal}, el filtro actúa
de manera conservadora: elimina el ruido ``sucio'' de la señal cruda
(línea negra) pero se adhiere perfectamente a las transiciones rápidas del
ojo (línea de color), asegurando que no se eliminen micro-movimientos
importantes ni se introduzca latencia artificial en la señal.

% --- [GRÁFICA: SEÑAL CRUDA VS FILTRADA EN 3 EJES] ---
\begin{figure}[H]
	\centering
	\includegraphics[width=1\textwidth]{Imagenes/senal_filtrada.png}
	\caption{Descomposición vectorial de la señal de mirada en un corto
		periodo de tiempo. Se compara la señal cruda (negro) con la señal
		filtrada (colores) para las componentes X, Y y Z del vector de mirada.
		El filtro Savitzky-Golay ($w=21,\ p=3$) elimina el ruido de alta
		frecuencia manteniendo la fidelidad de los cambios de posición bruscos
		(sacádicos).}
	\label{fig:filtrado_signal}
\end{figure}

Adicionalmente, el sistema mantuvo una estabilidad temporal rigurosa,
operando a una tasa de muestreo efectiva de 120 FPS. Esto permitió
reconstruir las trayectorias con una resolución temporal de 8.33 ms,
capturando la micro-estructura del movimiento que se perdería en cámaras
web convencionales de 30 o 60 Hz. Gao et al.\ \cite{gao2025eye_tracking_tutorial}
establecen que una frecuencia de muestreo mínima de 60 Hz es necesaria
para detectar sacádicos con fiabilidad, y que tasas superiores a 100 Hz
son recomendables cuando el objetivo es la extracción de características
cinemáticas de precisión como la aceleración o el \textit{jerk}; el
sistema implementado cumple ampliamente con ambos criterios.

A partir de la señal filtrada, se aplicó un umbral de velocidad angular
$v > 80~^\circ/s$ para clasificar automáticamente los eventos oculares en
dos categorías mutuamente excluyentes: sacádicos balísticos y períodos de
fijación estable. Este umbral es consistente con el criterio estándar
reportado por Engbert y Kliegl \cite{engbert2003microsaccades}, quienes
establecen que los movimientos sacádicos voluntarios superan invariablemente
este valor mientras que los micro-movimientos de fijación —incluyendo
microsácadas y deriva— permanecen por debajo de él en condiciones normales
de visión. La Figura~\ref{fig:deteccion_eventos} ilustra el resultado de
aplicar este criterio sobre la señal de un voluntario representativo.

\begin{figure}[H]
	\centering
	\includegraphics[width=0.95\textwidth]{Imagenes/eventos.png}
	\caption{Perfil temporal de la velocidad del vector de mirada para un
		voluntario representativo, obtenido tras la aplicación del filtro
		Savitzky-Golay. Los picos en rojo identifican eventos
		sacádicos y microsacádicos detectados mediante el umbral $v > 80~^\circ/s$
		\cite{engbert2003microsaccades}, y las franjas
		verdes delimitan los períodos de fijación
		estable entre movimientos balísticos consecutivos. Esta clasificación
		constituye la base para la extracción de las características biométricas
		analizadas en las secciones siguientes.}
	\label{fig:deteccion_eventos}
\end{figure}


\subsection{Precisión de la Detección (YOLOv8)}

El modelo de detección de pupila basado en YOLOv8n (Nano) demostró un rendimiento superior en comparación con los métodos clásicos. Tras el proceso de \textit{fine-tuning} con el dataset propio, se obtuvo una Precisión Media (mAP@50) de 99,5\%, con menos del 1\% de pérdidas de seguimiento (\textit{track-loss}) durante los parpadeos y movimientos rápidos.

\subsubsection{Métricas de Entrenamiento y Validación}

El entrenamiento se realizó durante \textbf{80 épocas} utilizando un tamaño de lote (\textit{batch size}) de 16 imágenes y optimización estocástica. La convergencia del modelo fue estable, estabilizando las pérdidas de caja (\textit{box\_loss}) y clasificación (\textit{cls\_loss}) en valores mínimos hacia la época 75.

La Tabla~\ref{tab:yolo_metrics} resume las métricas de rendimiento obtenidas en el conjunto de validación tras finalizar el entrenamiento.

\begin{table}[h!]
	\centering
	\caption{Métricas finales de rendimiento del modelo YOLOv8n tras 80 épocas de entrenamiento.}
	\vspace{0.2cm}
	\begin{tabular}{lc}
		\toprule
		\textbf{Métrica} & \textbf{Valor Obtenido} \\
		\midrule
		Precisión (Precision) & 0.999 \\
		Sensibilidad (Recall) & 1.000 \\
		mAP @ 0.50 & 0.995 \\
		mAP @ 0.50:0.95 & 0.912 \\
		Tiempo de Inferencia (promedio) & 24.7 ms \\
		\bottomrule
	\end{tabular}
	\label{tab:yolo_metrics}
\end{table}

Los resultados evidencian una robustez excepcional:
\begin{itemize}
	\item \textbf{Recall de 1.000}: Indica que el sistema fue capaz de detectar el 100\% de las pupilas presentes en el set de validación, confirmando la ausencia total de "falsos negativos" (pérdidas de tracking).
	\item \textbf{mAP@50-95 de 0.912}: Este valor, inusualmente alto para detección de objetos en tiempo real, demuestra que no solo se detecta la pupila, sino que el cuadro delimitador (\textit{bounding box}) se ajusta con alta precisión al contorno real del ojo, lo cual es crítico para el posterior cálculo del centroide.
\end{itemize}

\section{Caracterización Cinemática y Fisiológica}

Una vez validada la señal, se procedió a verificar que los movimientos registrados cumplen con las leyes fisiológicas conocidas del sistema oculomotor humano.

\subsection{Análisis de la Secuencia Principal (Main Sequence)}

La validación fisiológica de los movimientos capturados es un paso crítico para asegurar la integridad de los datos biométricos. Para ello, se analizó la relación entre la amplitud del movimiento sacádico ($A$) y su velocidad pico ($V_{pico}$), conocida como la \textit{Main Sequence}. Se procesaron los datos consolidados de la población completa ($N=15$), aplicando un umbral de detección estricto ($v > 80^\circ/s$) para aislar exclusivamente la dinámica balística y separar microsacádicos o ruido instrumental.

La Figura~\ref{fig:main_sequence} muestra la distribución de los sacádicos registrados junto con el ajuste del modelo exponencial teórico de Bahill, definido por la Ecuación~\ref{eq:bahill}.



\begin{figure}[H]
	\centering
	% Asegúrate de que el nombre del archivo coincida con el que guardaste
	\includegraphics[width=0.85\textwidth]{Imagenes/secuencia_principal.png}
	\caption{Diagrama de dispersión de la Secuencia Principal para la población completa ($N=15$). Los puntos representan los eventos oculares individuales detectados, mientras que la línea continua indica el ajuste del modelo exponencial de Bahill. La clara adherencia a la curva ($R^2 = 0.9216$) confirma que el sistema captura fielmente la saturación biomecánica del movimiento ocular humano.}
	\label{fig:main_sequence}
\end{figure}


\subsubsection{Discusión de Parámetros}

El ajuste de regresión no lineal sobre el corpus completo de movimientos sacádicos detectados ($n = 2{,}157$ eventos, provenientes de los $N = 15$ participantes) permitió extraer los parámetros del modelo exponencial de Bahill (Ecuación~\ref{eq:bahill}). La Tabla~\ref{tab:datos_descriptivos} resume las estadísticas descriptivas del conjunto de datos, y la Tabla~\ref{tab:mainseq_vs_literatura} contrasta los parámetros obtenidos con los rangos normativos consolidados en la literatura de oculomotricidad.

% --- Tabla 1: estadísticas descriptivas del corpus ---
\begin{table}[h!]
	\centering
	\caption{Estadísticas descriptivas del corpus de sacádicos utilizado para el ajuste de la \textit{Main Sequence} ($n = 2{,}157$ eventos, $N = 15$ sujetos).}
	\label{tab:datos_descriptivos}
	\vspace{0.2cm}
	\begin{tabular}{lcccccc}
		\toprule
		\textbf{Variable} & \textbf{Media} & \textbf{SD} & \textbf{Mediana} & \textbf{Mín.} & \textbf{Máx.} & \textbf{Unidad} \\
		\midrule
		Amplitud ($A$)        & 5.91  & 2.63  & 5.44  & 0.82  & 34.91 & $^\circ$ \\
		Velocidad pico ($V$)  & 198.48 & 64.92 & 189.38 & 65.14 & 584.43 & $^\circ$/s \\
		Residuo ($\hat{V}-V$) & 0.36  & 18.18 & ---   & ---   & ---   & $^\circ$/s \\
		\bottomrule
	\end{tabular}
\end{table}

% --- Tabla 2: comparación con literatura ---
\begin{table}[h!]
	\centering
	\caption{Comparación de los parámetros de la \textit{Main Sequence} obtenidos experimentalmente frente a los rangos normativos reportados para adultos sanos.}
	\label{tab:mainseq_vs_literatura}
	\vspace{0.2cm}
	\begin{tabular}{lccc}
		\toprule
		\textbf{Parámetro} & \textbf{Valor obtenido} & \textbf{Rango normativo} & \textbf{Referencia} \\
		\midrule
		$V_{sat}$ ($^\circ$/s) & 654.33 & 400 -- 800 & \cite{bahill1975, leigh2015neurology, collewijn1988binocular} \\
		$C$ ($^\circ$)          & 15.85  & 10 -- 20   & \cite{bahill1975, vanopstal1987skewness} \\
		$R^2$ del ajuste        & 0.9216 & $> 0.85$   & \cite{holmqvist2011eye} \\
		RMSE ($^\circ$/s)       & 18.18  & $< 30$     & \cite{leigh2015neurology} \\
		\bottomrule
	\end{tabular}
\end{table}

Todos los parámetros se encuentran dentro de los límites fisiológicos aceptados, lo que respalda la validez del sistema de captura. A continuación se analiza cada parámetro en profundidad.

% ===== Corpus de datos =====
\paragraph{Perfil de los datos recolectados}
La distribución de amplitudes refleja la naturaleza de la tarea visual: el $94.9\%$ de los sacádicos registrados son de rango corto ($A < 10^\circ$, $n = 2{,}047$), el $4.7\%$ son de rango medio ($10^\circ \leq A < 20^\circ$, $n = 102$) y únicamente el $0.4\%$ son de gran amplitud ($A \geq 20^\circ$, $n = 8$). Esta distribución sesgada es característica de tareas de exploración visual en pantalla \cite{holmqvist2011eye} y presenta las velocidades medias esperadas para cada régimen:

\begin{equation}
	\bar{V}_{A<10^\circ} = 189.7 \pm 51.9~^\circ/s, \quad
	\bar{V}_{10 \leq A < 20^\circ} = 348.5 \pm 43.0~^\circ/s, \quad
	\bar{V}_{A \geq 20^\circ} = 538.1 \pm 32.4~^\circ/s
	\label{eq:vel_por_grupo}
\end{equation}

El patrón de incremento de velocidad media con la amplitud ($189.7 \rightarrow 348.5 \rightarrow 538.1~^\circ/s$) confirma cualitativamente la relación exponencial de la \textit{Main Sequence} antes de cuantificarla formalmente mediante el ajuste paramétrico.

% ===== Vsat =====
\paragraph{Velocidad de saturación ($V_{sat} = 654.33~^\circ/s$).}

Bahill, Clark y Stark \cite{bahill1975} documentaron en el trabajo fundacional de la \textit{Main Sequence} que la velocidad de saturación en sácadas horizontales de adultos sanos se sitúa típicamente en el intervalo $[400, 700]~^\circ/s$. Collewijn, Erkelens y Steinman \cite{collewijn1988binocular}, empleando la técnica de alta precisión de bobina escleral, ampliaron dicho intervalo hasta los $800~^\circ/s$ en individuos jóvenes. Leigh y Zee \cite{leigh2015neurology}, en la referencia normativa canónica del campo, consolidan el rango $[400, 800]~^\circ/s$ como intervalo fisiológicamente esperable para movimientos sacádicos reflexivos en adultos neurológicamente sanos.

Para cuantificar la posición del valor obtenido dentro de este intervalo, se calcula el \textit{z-score} adoptando $\mu_{lit} = 600~^\circ/s$ (punto medio del rango normativo) y $\sigma_{lit} = 100~^\circ/s$ como estimador de la dispersión poblacional \cite{boghen1974velocity}:

\begin{equation}
	z = \frac{V_{sat} - \mu_{lit}}{\sigma_{lit}} = \frac{654.33 - 600}{100} = +0.543
	\label{eq:zscore_vsat}
\end{equation}

Un $z = 0.543$ indica que el valor obtenido se ubica a medio camino entre la media y el primer desvío estándar superior de la distribución normativa, lo que lo posiciona en el cuartil fisiológicamente más alto pero sin exceder los límites clínicamente aceptados. La desviación relativa respecto al punto central normativo es:

\begin{equation}
	\varepsilon_{rel} = \frac{|V_{sat} - \mu_{lit}|}{\mu_{lit}} \times 100\% = \frac{|654.33 - 600|}{600} \times 100\% \approx 9.06\%
	\label{eq:err_rel_vsat}
\end{equation}

Una desviación del $9.06\%$ es perfectamente compatible con la variabilidad inter-sujeto documentada en poblaciones de adultos jóvenes \cite{boghen1974velocity, collewijn1988binocular}, donde las diferencias individuales en el pico de velocidad pueden superar el $15\%$ sin constituir hallazgo patológico.

Desde el punto de vista del muestreo, este resultado valida el cumplimiento del teorema de Nyquist-Shannon \cite{nyquist1928certain}: para reconstruir fielmente una señal cuya componente espectral dominante sea $f_{bio}$, se requiere $f_s \geq 2 f_{bio}$. La dinámica sacádica, concentrada espectralmente por debajo de $50$ Hz \cite{harwood1999spectral}, impone un umbral de $f_s \geq 100$ Hz. La tasa empleada ($f_s = 120$ Hz) supera dicho umbral con un margen de seguridad del $20\%$, garantizando que la magnitud de $V_{sat}$ sea reconstruida sin atenuación por submuestreo.

% ===== C =====
\paragraph{Constante de amplitud ($C = 15.85^\circ$).}

El parámetro $C$ define la amplitud de transición entre el régimen lineal (velocidad proporcional a la amplitud) y el régimen de saturación neuromuscular. Matemáticamente, en $A = C$ la velocidad predicha por el modelo alcanza exactamente el $63.2\%$ de $V_{sat}$, en analogía con la constante de tiempo de un sistema de primer orden:

\begin{equation}
	V_{pico}(A = C) = V_{sat} \cdot \bigl(1 - e^{-1}\bigr) = 654.33 \times 0.6321 \approx 413.6~^\circ/s
	\label{eq:C_interpretation}
\end{equation}

Esto implica que para las sácadas de corta amplitud —que representan el $94.9\%$ del conjunto de datos ($\bar{A} = 5.91^\circ \ll C$)— el sistema opera en el régimen predominantemente lineal, donde la velocidad media observada ($198.48 \pm 64.92~^\circ/s$) refleja directamente la pendiente inicial de la curva y no el comportamiento de saturación.

Bahill et al.\ \cite{bahill1975} documentaron valores de $C$ en el intervalo $[10, 20]^\circ$ para adultos normales, mientras que Van Opstal y Van Gisbergen \cite{vanopstal1987skewness} indicaron que la asimetría del perfil de velocidad (\textit{skewness}) es máxima en la zona de transición definida por $C$, siendo sensible a la carga cognitiva y al estado de fatiga. La posición relativa del valor obtenido dentro del rango normativo es:

\begin{equation}
	P_{rel}(C) = \frac{C - C_{min}}{C_{max} - C_{min}} = \frac{15.85 - 10}{20 - 10} = 0.585
	\label{eq:C_relpos}
\end{equation}

Un $P_{rel} = 0.585$ sitúa el parámetro prácticamente en el centro del rango normativo ($58.5\%$ del intervalo), lo cual es indicativo de un control neuromuscular promedio sin tendencia hacia los extremos que podrían asociarse a patología (valores $< 10^\circ$ en parálisis del nervio oculomotor, valores $> 20^\circ$ en sacadomanía o déficit cerebeloso, según \cite{leigh2015neurology}).

% ===== Bondad de ajuste =====
\paragraph{Bondad de ajuste y análisis de residuos.}

El coeficiente de determinación $R^2 = 0.9216$ confirma que el modelo exponencial de Bahill explica el $92.16\%$ de la varianza en la velocidad pico, superando el umbral de $R^2 \geq 0.85$ establecido por Holmqvist et al.\ \cite{holmqvist2011eye} como criterio mínimo para aceptar que los datos provienen de un sistema oculomotor neurofisiológicamente normal. El análisis de los residuos del ajuste ($\hat{V} - V_{real}$) permite verificar además que no existe sesgo sistemático en el modelo:

\begin{equation}
	\bar{\varepsilon} = 0.36~^\circ/s, \quad
	\sigma_\varepsilon = 18.18~^\circ/s, \quad
	\text{RMSE} = 18.18~^\circ/s, \quad
	\text{MAE} = 15.08~^\circ/s
	\label{eq:residuos_stats}
\end{equation}

La media de los residuos es prácticamente nula ($\bar{\varepsilon} \approx 0$), lo que descarta la existencia de sesgo sistemático en la estimación. El error cuadrático medio ($\text{RMSE} = 18.18~^\circ/s$) representa únicamente el $2.77\%$ de $V_{sat}$, ratio que se mantiene dentro de los márgenes de error esperados para sistemas de eye-tracking basados en video \cite{holmqvist2011eye}. Adicionalmente, el $97.5\%$ de los residuos se encuentra dentro de la banda $\pm 2\sigma_\varepsilon$ ($\pm 36.36~^\circ/s$), comportamiento consistente con una distribución normal de los errores de ajuste, lo que valida los supuestos del método de regresión no lineal por mínimos cuadrados empleado.

\noindent\textbf{Síntesis:} Los tres criterios de validación se cumplen simultáneamente: (i) $V_{sat} \in [400, 800]~^\circ/s$, (ii) $C \in [10, 20]^\circ$, y (iii) $R^2 > 0.85$. Esta triple condición —y no cada criterio por separado— constituye la evidencia de que el sistema de captura registra actividad oculomotora genuina y no artefactos de detección o submuestreo, validando así la integridad de los datos para el posterior análisis biométrico.



\subsection{Perfiles de Velocidad y Jerk}

Para evaluar la calidad del control motor ocular a nivel microscópico, se analizaron los perfiles cinemáticos de movimientos sacádicos individuales. Esta evaluación es crítica para confirmar que el proceso de filtrado (Savitzky-Golay) eliminó el ruido instrumental sin distorsionar la dinámica natural del ojo.

La Figura~\ref{fig:perfil_velocidad} presenta la evolución temporal de la velocidad angular y el \textit{Jerk} (la derivada de la aceleración) para un movimiento sacádico representativo de $\approx 20^\circ$ de amplitud.

\begin{figure}[h!]
	\centering
	% Asegúrate de que el archivo coincida con el nombre que guardaste
	\includegraphics[width=0.9\textwidth]{Imagenes/vel_jerk.png}
	\caption{Perfil cinemático detallado de un sacádico horizontal. \textbf{Azul (Eje Izq):} Velocidad angular mostrando el perfil de campana esperado en un movimiento de aceleración y desaceleración \textbf{Naranja (Eje Der):} La señal de Jerk se mantiene acotada dentro de rangos fisiológicos, sin picos de ruido de alta frecuencia, lo que indica una reconstrucción estable de la trayectoria.}
	\label{fig:perfil_velocidad}
\end{figure}

El análisis de esta gráfica permite validar dos aspectos fundamentales:
\begin{itemize}
	\item \textbf{Suavidad de la Trayectoria:} La curva de velocidad es continua y suave, carente de las oscilaciones abruptas típicas del error de cuantificación digital. Esto demuestra que la resolución temporal de 120 FPS es suficiente para reconstruir la señal continua del movimiento.
	\item \textbf{Control Motor:} El perfil de Jerk (línea naranja) refleja el costo energético del movimiento. Al mantenerse controlado y sin ruido excesivo, confirma que las métricas derivadas de esta señal (como la eficiencia del movimiento) serán fiables para el análisis biométrico subsiguiente.
\end{itemize}

\section{Identificación de Patrones Biométricos}

Una vez validada la integridad física de la señal y la precisión del sistema de captura, se procede al núcleo de la investigación: la evaluación del movimiento ocular como huella biométrica única. En esta sección se presentan los hallazgos relacionados con la singularidad de los patrones oculares. Se parte de la hipótesis de que, aunque todos los humanos siguen la \textit{Main Sequence} (como se vio en la sección 4.2.1), la "micro-estrategia" que utiliza el cerebro de cada individuo para ejecutar esos movimientos (el nivel de Jerk, la complejidad fractal, la latencia pupilar) varía de forma consistente entre sujetos, permitiendo su diferenciación. A continuación, se analiza qué características específicas aportan mayor poder discriminativo al sistema.


% ===================================================================
% REEMPLAZO DE LAS DOS SUBSECCIONES EN Capitulo4.tex
% \subsection{Importancia de Características} y
% \subsection{Jerarquía de Relevancia Biométrica}
% ===================================================================

\subsection{Importancia de Características (\textit{Feature Importance})}

Para determinar qué variables aportan mayor poder discriminativo al
sistema, se entrenó el clasificador \textit{Random Forest} con los
hiperparámetros de la Tabla~\ref{tab:hiperparametros_rf} y se calculó
la importancia relativa de cada característica utilizando el criterio
de impureza de Gini \cite{breiman2001}. Formalmente, la importancia
de una característica $X_j$ se define como la reducción promedio
ponderada de impureza que produce a lo largo de todos los árboles
del ensemble:

\begin{equation}
	\text{Importancia}(X_j) = \frac{1}{|\mathcal{T}|}
	\sum_{t \in \mathcal{T}} \sum_{\substack{n \in t \\ X_j \text{ divide } n}}
	\frac{N_n}{N} \,\Delta\text{Gini}(n)
	\label{eq:gini_importance_cap4}
\end{equation}

donde $|\mathcal{T}| = 300$ es el número de árboles, $N_n$ es el número
de muestras que alcanzan el nodo $n$, $N$ es el total de muestras de
entrenamiento, y $\Delta\text{Gini}(n)$ es la reducción de impureza en
dicho nodo. Las importancias se normalizan de modo que sumen la unidad,
permitiendo su interpretación directa como fracciones del poder
discriminativo total del modelo. Los resultados, presentados en la
Figura~\ref{fig:feature_importance}, permiten contrastar empíricamente
la jerarquía teórica propuesta en la
Sección~\ref{subsec:justificacion_metricas} con el comportamiento real
del clasificador sobre los $n = 2{,}157$ sacádicos del corpus experimental.

\begin{figure}[h!]
	\centering
	\includegraphics[width=0.9\textwidth]{Imagenes/Importancia_Discriminativa.png}
	\caption{Ranking de importancia de Gini de las características
		biométricas. Las barras representan la fracción del poder
		discriminativo total aportada por cada variable al ensemble de
		300 árboles. Se observa un predominio de los descriptores
		morfológicos (\texttt{Pupil\_Mean}) sobre los cinemáticos,
		con una contribución secundaria pero significativa de
		\texttt{Main\_Seq\_Slope} y \texttt{Jerk\_Max},
		validando el carácter híbrido del sistema.}
	\label{fig:feature_importance}
\end{figure}

El análisis de la importancia de características arroja dos
conclusiones fundamentales:

\begin{enumerate}
	
	\item \textbf{Predominio de la Morfología (\texttt{Pupil\_Mean}):}
	La variable con mayor peso discriminativo resultó ser el diámetro
	pupilar promedio, confirmando la hipótesis planteada en la
	Categoría~II de la Sección~\ref{subsec:justificacion_metricas}.
	Este resultado es consistente con la literatura: Beatty
	\cite{beatty1982} demostró que el diámetro pupilar basal es un
	rasgo fisiológicamente estable modulado por el sistema nervioso
	autónomo, mientras que Duchowski \cite{duchowski2017} señala que
	las diferencias anatómicas en el tamaño del iris actúan como
	discriminador primario en sistemas de biometría ocular. Desde la
	perspectiva del criterio de Gini, esto implica que las particiones
	basadas en \texttt{Pupil\_Mean} producen nodos hijo con
	distribuciones de clase más puras que cualquier otra variable,
	reduciendo la impureza de Gini en mayor magnitud por unidad de
	umbral de corte \cite{breiman2001}. Fisiológicamente, el tamaño
	pupilar en reposo es un rasgo fenotípico estable cuya variabilidad
	inter-sujeto supera con amplitud a su variabilidad intra-sesión,
	condición necesaria para que cualquier descriptor sea biométricamente
	útil \cite{komogortsev2010}.
	
	\item \textbf{Contribución de la Dinámica
		(\texttt{Main\_Seq\_Slope}, \texttt{Jerk\_Max}):}
	Las métricas cinemáticas de la Categoría~I ocupan posiciones
	relevantes en el ranking, validando empíricamente que la
	variabilidad inter-sujeto en el control neuromuscular sacádico
	---documentada por Komogortsev et al.\ \cite{komogortsev2010}---
	es suficientemente consistente como para ser explotada por el
	clasificador. La pendiente de la Secuencia Principal
	(\texttt{Main\_Seq\_Slope}) refleja la eficiencia biomecánica
	individual del sistema sacádico \cite{bahill1975}; el
	\texttt{Jerk\_Max} captura el límite superior del esfuerzo
	neuromuscular durante el impulso balístico. Ambas variables
	aportan la capa de \textit{biometría conductual}: describen
	\textit{cómo} se mueve el ojo, complementando a los descriptores
	morfológicos que describen \textit{cómo es} el ojo. Las métricas
	con importancia de Gini cercana a cero fueron excluidas
	automáticamente del proceso de decisión, operando el ensemble
	como un selector de características implícito que penaliza la
	redundancia sin intervención manual \cite{breiman2001}.
	
\end{enumerate}

Esta combinación valida el diseño híbrido del sistema: los descriptores
morfológicos establecen una separación inicial robusta entre identidades,
mientras que los descriptores dinámicos refinan la clasificación en los
casos de solapamiento morfológico y añaden resistencia ante
suplantaciones, ya que el patrón neuromuscular individual es
considerablemente más difícil de reproducir artificialmente que
el tamaño de la pupila \cite{komogortsev2010, duchowski2017}.

% ===================================================================
\subsection{Jerarquía de Relevancia Biométrica}

Tras el análisis de importancias, se procedió a categorizar las métricas
seleccionadas por el modelo según su naturaleza física, con el objetivo
de establecer una interpretación fisiológica coherente de los factores
que facilitan la discriminación inter-sujeto. La
Tabla~\ref{tab:resumen_metricas_cap4} consolida esta taxonomía
junto con el mecanismo fisiológico que justifica el poder discriminativo
asignado por Gini a cada grupo.

\begin{table}[h!]
	\centering
	\caption{Taxonomía de métricas retenidas por el clasificador
		\textit{Random Forest} y su rol en la discriminación biométrica.
		Solo se incluyen las variables con importancia de Gini
		estadísticamente superior a cero tras el entrenamiento con
		$|\mathcal{T}|=300$ árboles.}
	\label{tab:resumen_metricas_cap4}
	\vspace{0.2cm}
	\begin{tabular}{@{}lp{3.8cm}p{5.5cm}l@{}}
		\toprule
		\textbf{Categoría} & \textbf{Métrica} &
		\textbf{Mecanismo fisiológico} & \textbf{Ref.} \\
		\midrule
		Morfología  &
		\texttt{Pupil\_Mean}, \texttt{Pupil\_CV} &
		Rasgo fenotípico estable; diferencias anatómicas en el músculo
		esfínter del iris. &
		\cite{beatty1982, duchowski2017} \\[4pt]
		
		Cinemática  &
		\texttt{Jerk\_Max}, \texttt{Acc\_Max} &
		Firma del control motor ocular; geometría muscular única
		del sistema oculomotor. &
		\cite{bahill1975, komogortsev2010} \\[4pt]
		
		Dinámica    &
		\texttt{Main\_Seq\_Slope} &
		Eficiencia neuromuscular del sacádico; pendiente de la
		relación velocidad-amplitud individual. &
		\cite{bahill1975} \\[4pt]
		
		Complejidad &
		\texttt{Fractal\_Dim} &
		Auto-similaridad de la señal de velocidad; estrategia
		cognitiva de exploración visual no lineal. &
		\cite{higuchi1988} \\
		\bottomrule
	\end{tabular}
\end{table}

La jerarquía observada es interpretable en términos del cociente
señal-ruido biométrico de cada categoría. Formalmente, para que
una métrica sea discriminativa debe satisfacer:

\begin{equation}
	\frac{\sigma^2_{inter}}{\sigma^2_{intra}} \gg 1
	\label{eq:snr_biometrico}
\end{equation}

donde $\sigma^2_{inter}$ es la varianza entre sujetos y
$\sigma^2_{intra}$ es la varianza dentro del mismo sujeto en
distintas sesiones \cite{komogortsev2010}. Los descriptores
morfológicos satisfacen esta condición con holgura porque las
diferencias anatómicas entre individuos (inter) son de orden
superior a las fluctuaciones fisiológicas intra-sesión (intra).
Los descriptores cinemáticos satisfacen la condición en menor
grado ---de ahí su posición secundaria en el ranking de Gini---
pero aportan información complementaria e irreducible: dos sujetos
con diámetros pupilares similares pueden diferir significativamente
en su \texttt{Jerk\_Max} o \texttt{Main\_Seq\_Slope}, haciendo que
la combinación de ambas categorías supere a cualquiera de ellas por
separado.

\noindent\textbf{Conclusión del análisis:}
La exactitud del sistema ($82.59\%$ para $k=15$ sujetos,
$\bar{F1} = 0.823 \pm 0.087$) no depende de una única variable
sino de la interacción entre los cuatro grupos de la
Tabla~\ref{tab:resumen_metricas_cap4}. Esta arquitectura de
características es consistente con el principio de complementariedad
biométrica \cite{duchowski2017}: los descriptores morfológicos
proveen la separación gruesa inicial, los cinemáticos refinan la
clasificación en regiones de solapamiento, y los de complejidad
no lineal capturan la micro-variabilidad conductual que los
descriptores lineales no pueden representar. El proceso de selección
automática por Gini garantiza además que el modelo no incorpore
características redundantes que incrementen la dimensionalidad del
espacio de decisión sin mejorar la separabilidad entre clases,
en línea con la recomendación metodológica de Breiman \cite{breiman2001}.


\subsection{Perfiles Biométricos Individuales}

Para visualizar las diferencias inter-sujeto de manera integral, se generaron gráficos de radar (\textit{Spider Plots}) que consolidan tanto las métricas cinemáticas como las morfológicas. Los datos fueron normalizados (escala 0-1) para permitir la comparación directa entre variables de distinta naturaleza física. La Figura~\ref{fig:radares} presenta los perfiles biométricos de tres participantes del estudio, evidenciando configuraciones estructurales claramente distinguibles.

\begin{figure}[H]
	\centering
	% Asegúrate de tener la imagen generada guardada como 'radares_comparativos.png'
	\includegraphics[width=1.0\textwidth]{Imagenes/radares_comparativos.png}
	\caption{Comparación de perfiles biométricos para tres participantes distintos. \textbf{(Izquierda)} El sujeto 1 muestra un perfil orientado a la dinámica (alta velocidad y tasa sacádica). \textbf{(Centro)}  El sujeto 2 se distingue por características anatómicas dominantes (mayor tamaño pupilar) y menor reactividad dinámica. \textbf{(Derecha)}  El sujeto 3 presenta un perfil balanceado con alta complejidad fractal. Estas "firmas visuales"  validan la hipótesis de unicidad del patrón oculomotor.}
	\label{fig:radares}
\end{figure}

El análisis cualitativo de estos perfiles revela que el sistema no depende de una sola variable para la identificación, sino de la interacción compleja entre ellas:
\begin{itemize}
	\item \textbf{Diversidad de Estrategias:} Mientras que algunos sujetos resuelven la tarea visual con movimientos rápidos y frecuentes (alta \textit{Tasa Sacádica}), otros adoptan estrategias más pausadas pero con mayor diámetro pupilar basal.
	\item \textbf{Complementariedad:} La forma poligonal resultante actúa como una huella digital multidimensional. Incluso si dos sujetos tuvieran velocidades similares, diferencias en su \textit{Jerk} o en su \textit{Dimensión Fractal} alterarían la geometría del gráfico, permitiendo su discriminación por parte de los algoritmos de clasificación.
\end{itemize}

\subsection{Visualización de Separabilidad (LDA)}

Para corroborar visualmente la capacidad del sistema para distinguir entre los 14 participantes, se aplicó un Análisis Discriminante Lineal (LDA) sobre el conjunto completo de métricas. Con el fin de garantizar la privacidad y neutralidad del análisis, los sujetos fueron codificados con etiquetas anónimas (P1-P14).

\begin{figure}[H]
	\centering
	% --- IMAGEN IZQUIERDA (2D) ---
	\begin{minipage}[b]{0.48\textwidth}
		\centering
		\includegraphics[width=\textwidth]{Imagenes/2d.png}
		\caption{Proyección en 2D (LD1 vs LD2)}
		\label{fig:lda_2d}
	\end{minipage}
	\hfill % Esto añade el espacio entre las dos imágenes
	% --- IMAGEN DERECHA (3D) ---
	\begin{minipage}[b]{0.48\textwidth}
		\centering
		\includegraphics[width=\textwidth]{Imagenes/3d.png}
		\caption{Proyección en 3D (LD1, LD2, LD3)}
		\label{fig:lda_3d}
	\end{minipage}
	\caption{Espacio de características transformado mediante LDA. Cada color (P1-P14) representa a un participante distinto. Se observa la formación de clústeres compactos y bien definidos, lo que confirma visualmente la separabilidad lineal de las identidades biométricas.}
	\label{fig:lda_analysis}
\end{figure}

Como se observa en la Figura~\ref{fig:lda_analysis}, los datos biométricos forman nubes de puntos claramente distinguibles:
\begin{itemize}
	\item \textbf{Eficacia de la Reducción:} Las tres primeras componentes discriminantes logran explicar el \textbf{85\%} de la varianza discriminatoria. Esto indica que la identidad oculomotora puede ser comprimida eficientemente sin perder información crítica.
	\item \textbf{Separabilidad:} Sujetos como P3 y P14 (verde y rosa en la gráfica 2D) que podrían solaparse en algunas métricas, quedan totalmente separados en el espacio 3D, demostrando la robustez del enfoque multidimensional.
\end{itemize}

\subsection{Evaluación de métodos alternativos: reducción dimensional y clustering}

Antes de abordar la clasificación supervisada, se exploraron diversas técnicas no supervisadas y de visualización con el objetivo de comprender la estructura subyacente del espacio biométrico y evaluar si era posible discriminar sujetos sin necesidad de entrenamiento con etiquetas. Los resultados de esta exploración se discuten a continuación.

\subsubsection{Análisis de Componentes Principales (PCA)}

El análisis de componentes principales se aplicó inicialmente para determinar la dimensionalidad intrínseca de los datos. La Figura~\ref{fig:pca_scree} muestra la varianza explicada acumulada en función del número de componentes. Se observa que son necesarios \textbf{10 componentes principales para alcanzar el 95\,\% de la varianza total}, lo que indica que el espacio biométrico es intrínsecamente de alta dimensión y no puede comprimirse drásticamente sin perder información relevante. Las proyecciones en 2D y 3D (Figura~\ref{fig:pca_2d_3d}) evidencian una cierta separación entre clases, pero con solapamientos significativos que impiden una discriminación fiable mediante inspección visual o métodos lineales simples.

\begin{figure}[H]
	\centering
	\includegraphics[width=0.7\textwidth]{Imagenes/varianza.png}
	\caption{Gráfico de sedimentación del análisis de componentes principales. Las barras representan la varianza explicada por cada componente individual, mientras que la línea roja muestra la varianza acumulada. Se requieren 10 componentes para alcanzar el 95\,\% de la varianza total, lo que evidencia la alta dimensionalidad intrínseca del espacio de características biométricas y la imposibilidad de reducirlo drásticamente sin pérdida significativa de información.}
	\label{fig:pca_scree}
\end{figure}

\begin{figure}[H]
	\centering
	\begin{minipage}[b]{0.48\textwidth}
		\centering
		\includegraphics[width=\textwidth]{Imagenes/pca_2d.png}
		\caption{Proyección de los datos en las dos primeras componentes principales (PC1 y PC2), que en conjunto explican aproximadamente el 20\,\% de la varianza total. Se observa un fuerte solapamiento entre las clases, lo que evidencia la limitada separabilidad lineal del espacio biométrico original.}
		\label{fig:pca_2d}
	\end{minipage}
	\hfill
	\begin{minipage}[b]{0.48\textwidth}
		\centering
		\includegraphics[width=\textwidth]{Imagenes/pca_3d.png}
		\caption{Proyección tridimensional de los datos sobre las tres primeras componentes principales (PC1, PC2 y PC3). Aunque se incorpora una dimensión adicional, el solapamiento entre sujetos persiste, confirmando que la varianza explicada por estas componentes es insuficiente para una discriminación fiable.}
		\label{fig:pca_3d}
	\end{minipage}
	\caption{Análisis de componentes principales (PCA) del espacio de características biométricas. Las proyecciones en 2D y 3D muestran la dificultad de separar linealmente a los 15 participantes, lo que justifica la necesidad de métodos no lineales o supervisados.}
	\label{fig:pca_2d_3d}
\end{figure}

\subsubsection{Visualización con t-SNE y UMAP}

Dado que PCA no lograba una separación nítida, se recurrió a técnicas de visualización no lineales: t-SNE y UMAP. Estas herramientas son excelentes para revelar estructuras locales en los datos, pero \textbf{no generan modelos que puedan aplicarse a nuevas muestras} (carecen de capacidad de generalización). Por lo tanto, su utilidad se limita a la exploración. En las Figuras~\ref{fig:tsne} y~\ref{fig:umap} se aprecia que, bajo ciertos hiperparámetros, los sujetos tienden a agruparse, aunque la forma de los grupos varía con la perplejidad (en t-SNE) o con los vecinos considerados (en UMAP). Estos resultados confirman que existe estructura agrupada, pero no proporcionan un método práctico para la identificación automática.

\begin{figure}[H]
	\centering
	\begin{minipage}[b]{0.48\textwidth}
		\centering
		\includegraphics[width=\textwidth]{Imagenes/tSNE.png}
		\caption{Visualización mediante t-SNE (perplejidad = 30) de las características biométricas.}
		\label{fig:tsne}
	\end{minipage}
	\hfill
	\begin{minipage}[b]{0.48\textwidth}
		\centering
		\includegraphics[width=\textwidth]{Imagenes/UMAP.png}
		\caption{Proyección mediante UMAP (n\_neighbors = 15, min\_dist = 0.1) del mismo espacio de características.}
		\label{fig:umap}
	\end{minipage}
	\caption{Visualización no lineal del espacio biométrico mediante t-SNE (izquierda) y UMAP (derecha). Ambas técnicas confirman la existencia de una estructura subyacente, pero al carecer de capacidad de generalización, su utilidad se limita a la exploración visual. Estos resultados refuerzan la necesidad de métodos supervisados como Random Forest para la identificación de sujetos.}
	\label{fig:tsne_umap}
\end{figure}

\subsubsection{Clustering no supervisado: KMeans y DBSCAN}

Para evaluar si los datos se agrupan naturalmente en las 15 identidades, se aplicaron dos algoritmos de clustering representativos.

\textbf{KMeans} (con \(k=15\)) arrojó un \textbf{ARI de 0.1334} y un \textbf{NMI de 0.2926} (Tabla~\ref{tab:resumen_clustering}). Estos valores, cercanos a cero, indican que la correspondencia entre los clústeres encontrados y las clases reales es prácticamente aleatoria. Además, la métrica de silueta (\(0.1295\)) sugiere que los clústeres no son compactos ni están bien separados, lo que descarta la hipótesis de que las identidades formen agrupaciones esféricas en el espacio de características.

\textbf{DBSCAN}, por su parte, produjo un resultado aún más revelador: con una parametrización que buscaba evitar el ruido excesivo, el algoritmo \textbf{agrupó el 79.6\,\% de las muestras como ruido} y el resto en un único clúster (ARI = 0.0). Esto demuestra que \textbf{no existe una estructura de densidad uniforme} que permita separar a los sujetos; las clases no están definidas por regiones de alta densidad separadas por zonas de baja densidad, sino que se entremezclan de forma compleja.

\begin{table}[H]
	\centering
	\caption{Métricas de rendimiento de los algoritmos de clustering no supervisado.}
	\label{tab:resumen_clustering}
	\begin{tabular}{lcccc}
		\toprule
		\textbf{Algoritmo} & \textbf{ARI} & \textbf{NMI} & \textbf{Silhouette} & \textbf{Observaciones} \\
		\midrule
		KMeans ($k=15$) & 0.1334 & 0.2926 & 0.1295 & --- \\
		DBSCAN          & 0.0000 & ---    & ---    & 79.6\,\% de puntos clasificados como ruido \\
		\bottomrule
	\end{tabular}
\end{table}

\subsubsection{Necesidad del aprendizaje supervisado}

Los resultados anteriores ponen de manifiesto que el problema de identificación biométrica mediante movimientos oculares no puede resolverse con técnicas puramente no supervisadas o de visualización. La alta dimensionalidad, la ausencia de clústeres esféricos y la falta de separación por densidad obligan a recurrir a \textbf{métodos supervisados capaces de aprender fronteras de decisión no lineales}. Es en este contexto donde el clasificador \textbf{Random Forest} emerge como la opción adecuada, como se detalla en la siguiente sección.

\section{Rendimiento de la Clasificación}

Para cuantificar la precisión del sistema como herramienta biométrica, se evaluaron dos clasificadores supervisados: Máquinas de Vectores de Soporte (SVM) y Bosques Aleatorios (Random Forest). El conjunto de datos fue dividido siguiendo una estrategia estratificada (80\% entrenamiento, 20\% prueba) para asegurar la representatividad de todas las clases.


\subsection{Métricas de los Modelos}

La Tabla~\ref{tab:resultados_clasificacion} resume el desempeño de los modelos evaluados en el conjunto de prueba.

\begin{table}[H]
	\centering
	\caption{Métricas de rendimiento de los clasificadores biométricos 
		en el conjunto de prueba ($k=15$ sujetos, $n_{test}=672$). Para el 
		\textit{Random Forest}, los valores de precisión, recall y F1 se 
		reportan como media $\pm$ desviación estándar entre clases.}
	\vspace{0.2cm}
	\begin{tabular}{lcccc}
		\toprule
		\textbf{Modelo} & \textbf{Accuracy} & \textbf{Precision} 
		& \textbf{Recall} & \textbf{F1-Score} \\
		\midrule
		SVM (Kernel RBF) & 76.34\% & 0.763 & 0.763 & 0.761 \\
		\textbf{Random Forest} & \textbf{82.59\%} 
		& $\mathbf{0.825 \pm 0.071}$ 
		& $\mathbf{0.826 \pm 0.116}$ 
		& $\mathbf{0.823 \pm 0.087}$ \\
		\bottomrule
	\end{tabular}
	\label{tab:resultados_clasificacion}
\end{table}

Los resultados indican que el clasificador \textbf{Random Forest} ofrece el mejor balance de rendimiento, alcanzando una exactitud global del \textbf{82.5\%}. Esta superioridad frente al SVM sugiere que las fronteras de decisión entre participantes son altamente no lineales y se benefician de la estructura jerárquica de los árboles de decisión, capaz de explotar mejor las interacciones complejas entre variables morfológicas y cinemáticas.


\subsection{Análisis de Confusión y Escalabilidad del Clasificador}

Para caracterizar el rendimiento real del sistema de identificación biométrica se evaluó el clasificador \textit{Random Forest} bajo tres configuraciones de escalabilidad ($k = 5$, $10$ y $15$ sujetos), con los hiperparámetros optimizados que se detallan en la Tabla~\ref{tab:hiperparametros_rf}. Este enfoque escalonado permite cuantificar cómo la complejidad del problema de clasificación crece con el número de identidades, y establecer la degradación esperada del rendimiento a medida que el espacio de clases se expande \cite{breiman2001}.

% --- Tabla de hiperparámetros ---
\begin{table}[h!]
	\centering
	\caption{Hiperparámetros del clasificador \textit{Random Forest} optimizados mediante búsqueda en rejilla (\textit{grid search}).}
	\label{tab:hiperparametros_rf}
	\vspace{0.2cm}
	\begin{tabular}{ll}
		\toprule
		\textbf{Hiperparámetro} & \textbf{Valor seleccionado} \\
		\midrule
		\texttt{n\_estimators}      & 300 \\
		\texttt{criterion}          & Gini \\
		\texttt{max\_depth}         & 50 \\
		\texttt{max\_features}      & $\sqrt{p}$ \\
		\texttt{min\_samples\_split} & 2 \\
		\texttt{min\_samples\_leaf}  & 1 \\
		\texttt{bootstrap}          & False \\
		\bottomrule
	\end{tabular}
\end{table}

La selección de $\sqrt{p}$ como número de características evaluadas en cada nodo (donde $p$ es el total de características del espacio) es la heurística empíricamente recomendada por Breiman \cite{breiman2001} para tareas de clasificación, ya que maximiza la decorrelación entre los árboles del ensemble sin sacrificar capacidad predictiva individual. El uso de \texttt{bootstrap = False} implica que cada árbol se entrena sobre la totalidad del conjunto de entrenamiento, estrategia que en conjuntos de tamaño moderado reduce el sesgo de la estimación a costa de una ligera pérdida de diversidad, lo cual es apropiado dado el número limitado de muestras por clase ($n_{clase} = 48$) \cite{breiman2001}.

% ===== Análisis de escalabilidad =====
\subsubsection{Análisis de Escalabilidad}

La Tabla~\ref{tab:escalabilidad} resume el rendimiento del clasificador en función del número de identidades registradas.

\begin{table}[h!]
	\centering
	\caption{Rendimiento del clasificador \textit{Random Forest} en función del número de sujetos ($k$). La confianza reportada corresponde a la probabilidad de clase promedio asignada por el ensemble de 300 árboles.}
	\label{tab:escalabilidad}
	\vspace{0.2cm}
	\begin{tabular}{ccccc}
		\toprule
		\textbf{Sujetos ($k$)} & \textbf{Muestras test} & \textbf{Exactitud} & \textbf{Conf. aciertos} & \textbf{Conf. errores} \\
		\midrule
		5  & 240 & 92.92\% & 0.821 & 0.621 \\
		10 & 480 & 92.29\% & 0.669 & 0.407 \\
		15 & 672 & 82.59\% & 0.621 & 0.383 \\
		\bottomrule
	\end{tabular}
\end{table}

Dos patrones cuantitativamente relevantes emergen de la Tabla~\ref{tab:escalabilidad}:

\textbf{(1) Degradación escalonada de la exactitud.} El sistema mantiene un rendimiento prácticamente estable al pasar de $k=5$ a $k=10$ sujetos (caída de apenas $0.63$ puntos porcentuales, $\Delta_{5\to10} = -0.63\%$), pero experimenta una caída más pronunciada al incorporar los cinco sujetos adicionales ($\Delta_{10\to15} = -9.70\%$). Este comportamiento es consistente con la teoría de clasificación multiclase: la probabilidad de confusión entre pares crece combinatoriamente con el número de clases ($\binom{k}{2}$ pares posibles: $10$, $45$ y $105$ para $k=5$, $10$ y $15$ respectivamente), de modo que la degradación no es lineal sino que se acelera a medida que el espacio de clases se satura \cite{komogortsev2010}.

\textbf{(2) Separabilidad de la confianza como indicador de calidad.} En los tres escenarios existe una diferencia estadísticamente significativa entre la confianza media de las predicciones correctas y la de los errores. Para $k=15$: $\bar{c}_{aciertos} = 0.621$ frente a $\bar{c}_{errores} = 0.383$, una brecha de $\Delta c = 0.238$. Este resultado indica que el ensemble no solo predice una clase, sino que asigna masas de probabilidad notablemente distintas cuando acierta versus cuando falla. Dicha propiedad es deseable en aplicaciones biométricas, donde un umbral de confianza mínimo puede implementarse para rechazar predicciones de baja certeza en lugar de emitir una identidad incorrecta \cite{duchowski2017}.

% ===== Análisis de confusión 15 sujetos =====
\subsubsection{Análisis de Confusión por Sujeto (k = 15)}

El modelo \textit{Random Forest} con los hiperparámetros de la Tabla~\ref{tab:hiperparametros_rf} fue evaluado sobre el conjunto de test balanceado ($48$ muestras por sujeto, $672$ en total). La Tabla~\ref{tab:metricas_por_sujeto} detalla las métricas por clase obtenidas.

\begin{table}[h!]
	\centering
	\caption{Métricas de clasificación por sujeto para el modelo \textit{Random Forest} ($k=15$). Precisión: fracción de predicciones de la clase que son correctas. Recall: fracción de instancias reales de la clase correctamente identificadas. F1: media armónica de ambas.}
	\label{tab:metricas_por_sujeto}
	\vspace{0.2cm}
	\begin{tabular}{lcccc}
		\toprule
		\textbf{Sujeto} & \textbf{Precisión} & \textbf{Recall} & \textbf{F1-score} & \textbf{$n_{test}$} \\
		\midrule
		P1  & 0.825 & 0.979 & 0.895 & 48 \\
		P2  & 0.959 & 0.979 & \textbf{0.969} & 48 \\
		P3  & 0.750 & 0.750 & 0.750 & 48 \\
		P4  & 0.938 & 0.938 & 0.938 & 48 \\
		P5  & 0.809 & 0.792 & 0.800 & 48 \\
		P6  & 0.700 & 0.583 & \textit{0.636} & 48 \\
		P7  & 0.741 & 0.833 & 0.784 & 48 \\
		P8  & 0.915 & 0.896 & 0.905 & 48 \\
		P9  & 0.700 & 0.729 & 0.714 & 48 \\
		P10 & 0.860 & 0.896 & 0.878 & 48 \\
		P11 & 0.846 & 0.917 & 0.880 & 48 \\
		P12 & 0.852 & 0.958 & 0.902 & 48 \\
		P13 & 0.816 & 0.646 & 0.721 & 48 \\
		P14 & 0.842 & 0.667 & 0.744 & 48 \\
		\midrule
		\textbf{Media} & \textbf{0.825 $\pm$ 0.071} & \textbf{0.826 $\pm$ 0.116} & \textbf{0.823 $\pm$ 0.087} & --- \\
		\bottomrule
	\end{tabular}
\end{table}

Las siguientes observaciones emergen del análisis por clase:

\textbf{Sujetos de alto rendimiento (F1 $\geq$ 0.90):} P2 ($F1 = 0.969$), P4 ($F1 = 0.938$), P8 ($F1 = 0.905$) y P12 ($F1 = 0.902$) son clasificados con alta fiabilidad. Estos sujetos presentan perfiles biométricos con alta varianza inter-clase relativa a su varianza intra-clase, lo que en términos del criterio de Gini implica particiones de nodo más puras y, consecuentemente, mayor importancia discriminativa de sus características en el ensemble \cite{breiman2001}.

\textbf{Sujetos de bajo rendimiento (F1 $<$ 0.75):} P6 ($F1 = 0.636$), P9 ($F1 = 0.714$) y P13 ($F1 = 0.721$) concentran la mayoría de los errores. En particular, P6 registró 20 clasificaciones incorrectas de 48 posibles ($\text{Recall} = 0.583$), siendo el sujeto más confundido del dataset. El hecho de que P9 sea el destino más frecuente de las predicciones erróneas (15 instancias de otros sujetos clasificadas como P9) sugiere que su perfil biométrico ocupa una región central del espacio de características, actuando como atractor de clases cercanas.

\textbf{Asimetría precisión-recall en P1, P13 y P14:} Estos sujetos presentan recall alto pero precisión moderada (P1: $P=0.825$, $R=0.979$), o bien el patrón inverso (P13: $P=0.816$, $R=0.646$). El primer caso indica que el clasificador tiende a asignar la clase a instancias de otros sujetos (falsos positivos); el segundo, que instancias reales del sujeto son atribuidas a otros (falsos negativos). Esta distinción es relevante en el contexto de seguridad biométrica: los falsos negativos implican denegación de acceso a usuarios legítimos, mientras que los falsos positivos implican acceso no autorizado, siendo estos últimos el escenario de mayor riesgo \cite{duchowski2017}.

\textbf{Consistencia estadística:} La desviación estándar del F1-score entre sujetos ($\sigma_{F1} = 0.087$) indica una dispersión moderada del rendimiento. El $95\%$ de los sujetos se espera en el intervalo $[0.823 - 2(0.087),\ 0.823 + 2(0.087)] = [0.649,\ 0.997]$, rango que efectivamente contiene a todos los sujetos de la Tabla~\ref{tab:metricas_por_sujeto}.

% ===== Figuras de matrices =====
\subsubsection{Matrices de Confusión por Configuración}

Las Figuras~\ref{fig:matriz_confusion2}, \ref{fig:matriz_confusion3} y \ref{fig:matriz_confusion} presentan las matrices de confusión normalizadas para $k = 5$, $10$ y $15$ sujetos respectivamente. La normalización por fila transforma cada celda $(i,j)$ en la probabilidad condicional $P(\hat{y}=j \mid y=i)$, facilitando la interpretación de la tasa de error específica por clase independientemente del número total de muestras \cite{holmqvist2011eye}.

\begin{figure}[H]
	\centering
	\includegraphics[width=1\textwidth]{Imagenes/matriz_5_sujetos.png}
	\caption{Matriz de confusión normalizada — \textit{Random Forest} con $k=5$ sujetos (exactitud: $92.92\%$, confianza media en aciertos: $0.821$). La diagonal prácticamente unitaria refleja la alta separabilidad del espacio de características cuando el número de clases es reducido.}
	\label{fig:matriz_confusion2}
\end{figure}

\begin{figure}[H]
	\centering
	\includegraphics[width=1\textwidth]{Imagenes/matriz_10_sujetos.png}
	\caption{Matriz de confusión normalizada — \textit{Random Forest} con $k=10$ sujetos (exactitud: $92.29\%$, confianza media en aciertos: $0.669$). La exactitud se mantiene estable respecto a $k=5$ a pesar de duplicar el número de clases, evidenciando la robustez del sistema en el régimen de escalabilidad moderada.}
	\label{fig:matriz_confusion3}
\end{figure}

\begin{figure}[H]
	\centering
	\includegraphics[width=1\textwidth]{Imagenes/matriz.png}
	\caption{Matriz de confusión normalizada — \textit{Random Forest} con $k=15$ sujetos (exactitud: $82.59\%$, $\bar{F1} = 0.823 \pm 0.087$). La diagonal dominante confirma el alto rendimiento global del sistema. Las confusiones fuera de la diagonal se concentran en P6 y P9, sujetos cuyo perfil biométrico presenta mayor solapamiento inter-clase.}
	\label{fig:matriz_confusion}
\end{figure}

En conjunto, la progresión de resultados ($92.92\% \to 92.29\% \to 82.59\%$) es coherente con los valores reportados en sistemas de biometría oculomotora de la literatura: Komogortsev et al.\ \cite{komogortsev2010} documentan tasas de identificación correcta entre el $70\%$ y el $95\%$ para cohortes de 10 a 20 sujetos usando modelos del planta oculomotora, validando que el rendimiento obtenido es competitivo incluso considerando que el sistema propuesto opera con hardware de bajo costo frente a los eye-trackers de precisión utilizados en dichos estudios.

\section{Evaluación para Control de Cursor}

Para validar la utilidad práctica del vector de mirada estimado en aplicaciones de Interacción Humano-Computadora (HCI), se analizó su desempeño en una tarea de apuntamiento visual. La propuesta de interacción se diseñó bajo un esquema intuitivo: el vector de mirada controla la posición espacial ($X, Y$) del cursor en tiempo real, mientras que el gesto de parpadeo voluntario (detectado por la pérdida momentánea de la pupila) se traduce como el evento de selección o "clic".

\subsection{Mapeo y Corrección (Matriz de Homografía)}

Dado que el vector de mirada en el espacio 3D no se traduce linealmente a coordenadas de píxeles en pantalla (debido a la posición relativa de la cámara y la distorsión de lente), se implementó una etapa de calibración mediante una transformación proyectiva. Se calculó una Matriz de Homografía ($H$) que mapea las proyecciones del vector de mirada ($g_x, g_y$) a las coordenadas de pantalla ($s_x, s_y$):

\begin{equation}
	\begin{bmatrix} s_x \\ s_y \\ 1 \end{bmatrix} = 
	H \cdot 
	\begin{bmatrix} g_x \\ g_y \\ 1 \end{bmatrix}
\end{equation}

Es fundamental destacar que el funcionamiento adecuado del cursor depende críticamente de la precisión de esta matriz $H$. Cualquier desviación durante la fase de calibración (por movimientos de cabeza del usuario o falta de atención a los puntos guía) introduce un error sistemático en la proyección, degradando la experiencia de control.

\begin{figure}[h!]
	\centering
	\includegraphics[width=0.85\textwidth]{Imagenes/correccion_homografica.png}
	\caption{Distribución espacial de las coordenadas oculares proyectadas mediante un mapeo estrictamente lineal. La ausencia de corrección homográfica evidencia el error de paralaje y la deformación geométrica entre el plano de la cámara y el plano de la pantalla.}
	\label{fig:heatmap_mapeado_lineal}
\end{figure}

% ===================================================================
% REEMPLAZO COMPLETO DE: \subsection{Precisión Espacial y Estabilidad}
% en Capitulo4.tex
% ===================================================================

\subsection{Precisión Espacial y Estabilidad}

Para evaluar la precisión del sistema de mapeo en condiciones de uso
variadas, se diseñaron tres protocolos de validación con distinta
complejidad espacial y temporal: una cuadrícula de $3\times3$ puntos
(9 objetivos estáticos), una cuadrícula reducida de 5 puntos, y un
estímulo de seguimiento continuo en trayectoria espiral. Esta batería
de pruebas permite caracterizar el sistema tanto en el régimen de
\textit{fijación discreta} ---donde el usuario dirige la mirada hacia
objetivos estáticos--- como en el régimen de \textit{seguimiento suave}
(\textit{smooth pursuit}), que exige un control oculomotor
cualitativamente distinto \cite{leigh2015neurology}.

% ===== EXPERIMENTO 1: 9 puntos =====
\subsubsection{Experimento I: Cuadrícula $3\times3$ (9 Puntos)}

La Figura~\ref{fig:heatmap_cursor} presenta el mapa de calor acumulado
durante la prueba de 9 puntos. Se observan claramente 9 clústeres de
densidad que corresponden a los puntos de estímulo, con las zonas de
mayor concentración (rojo) alineadas con las posiciones objetivo y las
trayectorias sacádicas visibles como líneas tenues entre clústeres.

\begin{figure}[h!]
	\centering
	\includegraphics[width=0.85\textwidth]{Imagenes/hm.png}
	\caption{Mapa de calor de la mirada corregida mediante la matriz de
		homografía para la cuadrícula $3\times3$. Las zonas rojas (alta
		densidad) coinciden con la ubicación de los 9 puntos de calibración,
		demostrando que el usuario mantuvo fijaciones estables sobre los
		objetivos. Las líneas tenues entre clústeres representan las
		trayectorias sacádicas de transición.}
	\label{fig:heatmap_cursor}
\end{figure}

El análisis cuantitativo de este experimento arroja los siguientes
indicadores:

\begin{itemize}
	\item \textbf{Precisión (Error medio):} El error promedio se situó
	en $\bar{e}_9 = \pm 27$ píxeles sobre una pantalla de
	$1920 \times 1080$ píxeles. Expresado como fracción del ancho de
	pantalla:
	\begin{equation}
		\varepsilon_9 = \frac{27}{1920} \times 100\% \approx 1.41\%
		\label{eq:error_9puntos}
	\end{equation}
	Este margen es aceptable para interfaces de accesibilidad con
	elementos de interacción de tamaño $\geq 54$ píxeles
	(equivalente a $2\bar{e}_9$), pero limita la interacción con
	elementos pequeños como hipervínculos de texto \cite{duchowski2017}.
	
	\item \textbf{Estabilidad (\textit{Jitter}):} El filtrado temporal
	mediante Savitzky-Golay redujo la vibración del cursor durante las
	fijaciones, permitiendo periodos de reposo visualmente estables
	sobre cada objetivo.
\end{itemize}

% ===== EXPERIMENTO 2: 5 puntos =====
\subsubsection{Experimento II: Cuadrícula Reducida (5 Puntos)}

Para evaluar si la reducción del número de puntos de calibración afecta
la precisión del mapeo, se repitió el protocolo con una cuadrícula de
5 objetivos (esquinas más centro). La Figura~\ref{fig:heatmap_5_puntos}
presenta el mapa de calor resultante, donde se identifican los 5
clústeres de fijación correspondientes.

\begin{figure}[h!]
	\centering
	\includegraphics[width=0.85\textwidth]{Imagenes/hm_5_puntos.png}
	\caption{Mapa de calor de la mirada para la cuadrícula de 5 puntos.
		Los 5 clústeres de densidad coinciden con las posiciones objetivo.
		La menor densidad de puntos de referencia respecto al experimento
		de 9 puntos se traduce en una menor restricción geométrica de la
		matriz de homografía, lo que incrementa el error de proyección
		en las regiones intermedias de la pantalla.}
	\label{fig:heatmap_5_puntos}
\end{figure}

El error promedio en este experimento fue $\bar{e}_5 = \pm 75$ píxeles,
equivalente al $3.91\%$ del ancho de pantalla:

\begin{equation}
	\varepsilon_5 = \frac{75}{1920} \times 100\% \approx 3.91\%
	\label{eq:error_5puntos}
\end{equation}

La degradación respecto al experimento de 9 puntos es cuantificable
como:

\begin{equation}
	\Delta\varepsilon = \frac{\bar{e}_5 - \bar{e}_9}{\bar{e}_9}
	\times 100\% = \frac{75 - 27}{27} \times 100\% \approx 177.8\%
	\label{eq:degradacion_5_9}
\end{equation}

Un incremento del error de casi el $178\%$ al reducir los puntos de
calibración de 9 a 5 evidencia que la densidad de puntos de referencia
es crítica para la calidad de la transformación proyectiva. Desde el
punto de vista matemático, la matriz de homografía $H$ se estima
mediante mínimos cuadrados sobre $k$ correspondencias de puntos: a mayor
$k$, el sistema está más sobredeterminado y la estimación de $H$ es más
robusta ante el ruido de medición \cite{duchowski2017}. Con solo 5
puntos, el sistema queda en el límite mínimo para una homografía
($k_{min} = 4$ pares no colineales), lo que reduce significativamente
la capacidad del modelo para compensar la distorsión no lineal de la
óptica de la cámara en las regiones intermedias de la pantalla.

% ===== EXPERIMENTO 3: Espiral =====
\subsubsection{Experimento III: Seguimiento de Trayectoria Espiral}

El tercer protocolo evaluó la capacidad del sistema para registrar
movimientos oculares de \textit{seguimiento suave} (\textit{smooth
	pursuit}), en los que el participante rastreó con la mirada un punto
animado cuya trayectoria describía una espiral de Arquímedes sobre la
pantalla. Este tipo de movimiento es cualitativamente diferente a las
sacadas discretas de los experimentos anteriores: involucra el sistema
de seguimiento de velocidad del cerebelo y produce perfiles de velocidad
continuos en lugar de los impulsos balísticos de las sacadas
\cite{leigh2015neurology}.

La Figura~\ref{fig:mapeo_espiral} superpone la trayectoria teórica de
la espiral (curva de referencia) con la trayectoria registrada por el
sistema (mirada proyectada en pantalla), permitiendo evaluar visualmente
la fidelidad del seguimiento.

\begin{figure}[h!]
	\centering
	\includegraphics[width=0.85\textwidth]{Imagenes/mapeo_espiral.png}
	\caption{Superposición de la trayectoria espiral teórica (curva de
		referencia) y la trayectoria de mirada registrada por el sistema.
		La adherencia de la mirada a la curva teórica evalúa la capacidad
		del sistema para reconstruir movimientos oculares continuos de
		seguimiento suave (\textit{smooth pursuit}), en contraste con las
		sacadas discretas de los experimentos anteriores.}
	\label{fig:mapeo_espiral}
\end{figure}

La desviación media de la trayectoria registrada respecto a la espiral
teórica fue $\bar{e}_{esp} = 35$ píxeles, equivalente al $1.82\%$ del
ancho de pantalla:

\begin{equation}
	\varepsilon_{esp} = \frac{35}{1920} \times 100\% \approx 1.82\%
	\label{eq:error_espiral}
\end{equation}

Este resultado es notable por dos razones. En primer lugar, el error de
seguimiento de la espiral es considerablemente menor que el del
experimento de 5 puntos ($35$ vs $75$ px), a pesar de que la trayectoria
espiral impone una demanda de control motor significativamente mayor.
Esto confirma que el factor limitante en el experimento de 5 puntos no
fue la capacidad del sistema oculomotor del participante sino la menor
calidad de la estimación de $H$ por escasez de puntos de referencia.
En segundo lugar, el error de seguimiento es solo marginalmente superior
al del experimento de 9 puntos ($35$ vs $27$ px), siendo la diferencia:

\begin{equation}
	\Delta e_{esp-9} = \frac{\bar{e}_{esp} - \bar{e}_9}{\bar{e}_9}
	\times 100\% = \frac{35 - 27}{27} \times 100\% \approx 29.6\%
	\label{eq:degradacion_esp_9}
\end{equation}

Un incremento del $29.6\%$ en el error al pasar de fijaciones discretas
a seguimiento continuo es coherente con los valores reportados en la
literatura para sistemas de eye-tracking basados en video: Holmqvist
et al.\ \cite{holmqvist2011eye} documentan que el error de seguimiento
suave suele ser entre un $20\%$ y un $40\%$ superior al error de
fijación estática en sistemas de resolución temporal comparable, debido
a la latencia inherente del pipeline de procesamiento de imagen.

% ===== RESUMEN COMPARATIVO =====
\subsubsection{Síntesis Comparativa de los Tres Experimentos}

La Tabla~\ref{tab:comparativa_experimentos} consolida los resultados
de precisión de los tres protocolos de validación.

\begin{table}[h!]
	\centering
	\caption{Comparativa de precisión espacial entre los tres protocolos
		de validación del sistema de control de cursor.}
	\label{tab:comparativa_experimentos}
	\vspace{0.2cm}
	\begin{tabular}{lcccl}
		\toprule
		\textbf{Experimento} & \textbf{Tipo} &
		\textbf{Error (px)} & \textbf{Error (\%)} &
		\textbf{Factor limitante} \\
		\midrule
		9 puntos & Fijación discreta & $\pm$27 & 1.41\% &
		Línea base del sistema \\
		5 puntos & Fijación discreta & $\pm$75 & 3.91\% &
		Calidad de homografía \\
		Espiral  & Seguimiento suave & $\pm$35 & 1.82\% &
		Latencia del pipeline \\
		\bottomrule
	\end{tabular}
\end{table}

Los resultados de la Tabla~\ref{tab:comparativa_experimentos} permiten
extraer dos conclusiones de diseño. Primero, la densidad de puntos de
calibración es el factor de mayor impacto sobre la precisión del sistema:
pasar de 9 a 5 puntos de referencia triplica el error. Segundo, el
sistema es capaz de seguir trayectorias continuas complejas con una
precisión comparable a la de las fijaciones estáticas bien calibradas,
lo que abre la posibilidad de aplicaciones que vayan más allá del control
por fijación puntual, como el trazado de rutas o la
interacción con interfaces gráficas de forma libre.


\subsection{Viabilidad y Trabajo Futuro}

Esta implementación constituye una prueba de concepto. Si bien se demostró la viabilidad técnica de controlar el cursor y ejecutar clics mediante parpadeos, la fluidez de la interacción es subjetiva y altamente sensible a la calidad de la calibración inicial. Esta sección abre la puerta a futuras investigaciones enfocadas en algoritmos de calibración dinámica o corrección no lineal que mejoren la robustez del sistema ante movimientos naturales de la cabeza.


\section{Análisis de Robustez Temporal y Deriva Biométrica}

La estabilidad de una firma biométrica ante variables fisiológicas y
ambientales no controladas es un requisito fundamental para su viabilidad
en aplicaciones de seguridad reales. Komogortsev et al.\
\cite{komogortsev2010} señalan que la variabilidad intra-sujeto
inter-sesión es el principal factor limitante en sistemas de biometría
oculomotora, y que dicha variabilidad se amplifica significativamente
bajo condiciones de fatiga o cambios de iluminación. Para cuantificar
este efecto en el sistema propuesto, se diseñó un experimento
cronobiológico comparando al sujeto P11 en dos instancias temporales:

\begin{enumerate}
	\item \textbf{Sesión Matutina (Línea Base):} 10:00 AM, bajo
	iluminación natural y tras descanso nocturno completo ($n=479$
	muestras).
	\item \textbf{Sesión Vespertina (Condición de Estrés):} 06:00 PM,
	tras una jornada laboral completa y bajo iluminación artificial
	($n=320$ muestras).
\end{enumerate}

Adicionalmente, se incorporó el perfil del sujeto P5 como referencia de
confusión ($n=159$ muestras), por ser el sujeto hacia el cual el
clasificador tiende a desplazar erróneamente las predicciones de P11
en la sesión vespertina. El análisis abarca ocho métricas biométricas
de las cuatro categorías definidas en la
Sección~\ref{subsec:justificacion_metricas}.

\subsection{Degradación del Rendimiento por Fatiga}

Los resultados de la clasificación, ilustrados en la
Figura~\ref{fig:comparativa_temporal}, evidencian una discrepancia notable
en el desempeño del sistema según la hora del día.

\begin{figure}[h!]
	\centering
	\includegraphics[width=1.0\textwidth]{Imagenes/COMPARATIVA_FINAL_MATRICES.png}
	\caption{Comparativa de robustez temporal para el sujeto P11.
		\textbf{Izquierda (Mañana):} la densidad de aciertos se concentra
		en la diagonal principal, indicando alta precisión. \textbf{Derecha
			(Tarde):} el desplazamiento sistemático de predicciones hacia la
		columna de P5 refleja la convergencia biométrica selectiva inducida
		por el cambio de iluminación.}
	\label{fig:comparativa_temporal}
\end{figure}

En la sesión matutina, las predicciones se concentran en la diagonal
principal de la matriz. En la sesión vespertina, en cambio, la distribución
de predicciones se desplaza sistemáticamente hacia P5, sugiriendo que la
identidad biométrica de P11 no se destruye globalmente sino que se
\textit{desplaza selectivamente} en una dimensión específica del espacio
de características.

\subsection{Análisis Estadístico de la Deriva: Ocho Métricas}
Para identificar el mecanismo específico de la confusión, se analizaron
ocho métricas de las cuatro categorías biométricas. Para cada métrica
se calcularon: el cambio porcentual entre sesiones ($\Delta\%$), la
significancia estadística mediante la prueba de Mann-Whitney-U
(no paramétrica, adecuada dado que no se asume normalidad), el tamaño
del efecto mediante la $d$ de Cohen:

\begin{equation}
	d = \frac{\mu_1 - \mu_2}{\sqrt{\frac{s_1^2 + s_2^2}{2}}}
	\label{eq:cohen_d}
\end{equation}

\noindent donde $\mu_1$ y $\mu_2$ son las medias de cada sesión y $s_1$, $s_2$
sus desviaciones estándar respectivas; y el \textbf{coeficiente de convergencia} $\lambda$:

\begin{equation}
	\lambda_j = \frac{\bar{X}_{j,\text{tarde}} - \bar{X}_{j,\text{mañ}}}
	{\bar{X}_{j,P5} - \bar{X}_{j,\text{mañ}}}
	\label{eq:lambda_convergencia}
\end{equation}

donde $\bar{X}_{j,c}$ es la media de la métrica $j$ bajo la condición
$c$. Un valor $\lambda_j > 0$ indica que la fatiga \textit{acerca} a P11
hacia P5 en esa dimensión (\textit{convergencia}); $\lambda_j < 0$ indica
que los \textit{aleja} (\textit{divergencia}); $\lambda_j = 1$ implica
convergencia completa. La Tabla~\ref{tab:deriva_completa} consolida
los resultados de las ocho métricas.

\begin{table}[h!]
	\centering
	\caption{Análisis estadístico de la deriva biométrica para ocho
		métricas. Prueba de Mann-Whitney-U (M$\to$T: Mañana vs Tarde).
		$d$: $d$ de Cohen. $\lambda$: coeficiente de convergencia hacia P5
		(Ec.~\ref{eq:lambda_convergencia}). En negrita: efectos de mayor
		relevancia clínica o clasificatoria.}
	\label{tab:deriva_completa}
	\vspace{0.2cm}
	\resizebox{\textwidth}{!}{%
		\begin{tabular}{lrrrrrrc}
			\toprule
			\textbf{Métrica} &
			\textbf{Mañana} &
			\textbf{Tarde} &
			\textbf{P5} &
			\textbf{$\Delta$\% M$\to$T} &
			\textbf{$p$ (M$\to$T)} &
			\textbf{$d$ (M$\to$T)} &
			\textbf{$\lambda$} \\
			\midrule
			\texttt{Pupil\_Mean}
			& $58.73 \pm 11.08$
			& $75.55 \pm 9.13$
			& $80.05 \pm 6.68$
			& $\mathbf{+28.64\%}$
			& $< 0.001$
			& $\mathbf{-1.626}$
			& $\mathbf{+0.789}$ \\
			
			\texttt{Pupil\_CV}
			& $0.045 \pm 0.023$
			& $0.056 \pm 0.034$
			& $0.037 \pm 0.015$
			& $+23.30\%$
			& $< 0.001$
			& $-0.384$
			& $-1.188$ \\
			
			\texttt{Pupil\_Vel\_Max}
			& $240.0 \pm 415.9$
			& $258.7 \pm 230.7$
			& $299.1 \pm 171.5$
			& $+7.79\%$
			& $< 0.001$
			& $-0.053$
			& $+0.316$ \\
			
			\texttt{Vel\_Mean}
			& $26.98 \pm 8.05$
			& $26.09 \pm 7.85$
			& $28.78 \pm 9.81$
			& $-3.29\%$
			& $0.049$
			& $0.111$
			& $-0.494$ \\
			
			\texttt{Jerk\_Mean}
			& $11{,}419 \pm 3{,}249$
			& $12{,}646 \pm 2{,}929$
			& $11{,}025 \pm 3{,}706$
			& $+10.75\%$
			& $< 0.001$
			& $-0.393$
			& $-3.113$ \\
			
			\texttt{Jerk\_Max}
			& $129{,}549 \pm 32{,}638$
			& $144{,}029 \pm 19{,}920$
			& $123{,}419 \pm 30{,}327$
			& $+11.18\%$
			& $< 0.001$
			& $\mathbf{-0.513}$
			& $-2.362$ \\
			
			\texttt{Main\_Seq\_Slope}
			& $1{,}187 \pm 336$
			& $1{,}183 \pm 308$
			& $1{,}041 \pm 310$
			& $-0.35\%$
			& $0.289$
			& $0.013$
			& $+0.029$ \\
			
			\texttt{Fractal\_Dim}
			& $1.111 \pm 0.026$
			& $1.098 \pm 0.022$
			& $1.119 \pm 0.026$
			& $-1.13\%$
			& $< 0.001$
			& $0.514$
			& $-1.572$ \\
			\bottomrule
	\end{tabular}}
\end{table}


\subsection{Mecanismo de Confusión: Convergencia Unidimensional}

El hallazgo central de la Tabla~\ref{tab:deriva_completa} es contraintuitivo
y de alta relevancia para el diseño del sistema: \textbf{la confusión del
	clasificador no es producto de una convergencia multidimensional general,
	sino de una convergencia unidimensional concentrada en \texttt{Pupil\_Mean},
	la variable de mayor peso de Gini del modelo}.

\paragraph{Convergencia en \texttt{Pupil\_Mean} ($\lambda = +0.789$).}
Esta es la única métrica que converge sustancialmente hacia P5. El
diámetro pupilar de P11 se incrementa un $\mathbf{+28.64\%}$ entre
sesiones ($58.73 \to 75.55$ unidades), con un efecto de Cohen de
$d = -1.626$ (\textbf{grande}), el mayor de todas las métricas
analizadas. Este aumento corresponde a la midriasis inducida por la
reducción de iluminación ambiental: Beatty \cite{beatty1982} demostró
que el diámetro pupilar en reposo es inversamente proporcional a la
iluminancia del entorno. El resultado es que P11-tarde se sitúa al
$78.9\%$ del camino hacia el perfil de P5 en esta dimensión,
expresado formalmente como:

\begin{equation}
	\lambda_{Pupil} =
	\frac{\bar{X}_{Pupil,tarde} - \bar{X}_{Pupil,man}}
	{\bar{X}_{Pupil,P5}   - \bar{X}_{Pupil,man}}
	= \frac{75.55 - 58.73}{80.05 - 58.73} = \frac{16.82}{21.32}
	\approx 0.789
	\label{eq:lambda_pupil_num}
\end{equation}

Dado que \texttt{Pupil\_Mean} es la característica con mayor importancia
de Gini en el modelo (Figura~\ref{fig:feature_importance}), esta
convergencia unidimensional es suficiente para dominar la decisión del
ensemble y producir clasificaciones erróneas hacia P5.

\paragraph{Divergencia activa en las métricas cinemáticas.}
El aspecto más relevante del análisis es que todas las métricas
cinemáticas presentan \textbf{divergencia} ($\lambda < 0$), es decir,
la fatiga aleja a P11 de P5 en esas dimensiones:

\begin{itemize}
	\item \texttt{Jerk\_Mean}: $\lambda = -3.113$ — el Jerk medio de P11
	aumenta un $+10.75\%$ en la tarde ($d = -0.393$, efecto pequeño),
	alejándose del perfil de P5 en más de tres veces la distancia
	original. Leigh y Zee \cite{leigh2015neurology} atribuyen este
	aumento a la irregularidad en la fase de desaceleración sacádica
	bajo fatiga.
	
	\item \texttt{Jerk\_Max}: $\lambda = -2.362$, $d = -0.513$ (mediano)
	— el Jerk máximo aumenta un $+11.18\%$, alejándose aún más de P5
	cuyo Jerk máximo es $\mathbf{15.4\%}$ menor ($123{,}419$ vs
	$144{,}029$). La diferencia tarde$\to$P5 es grande ($d = 0.863$,
	$p < 0.001$), lo que en condiciones ideales debería \textit{ayudar}
	al clasificador a separar a P11.
	
	\item \texttt{Fractal\_Dim}: $\lambda = -1.572$, $d = 0.514$
	(mediano) — la dimensión fractal de P11 disminuye levemente
	($-1.13\%$) pero la diferencia con P5 se amplifica
	($d_{T\to P5} = -0.898$, efecto grande), indicando que esta
	métrica diverge activamente de P5 con la fatiga.
\end{itemize}

\paragraph{\texttt{Main\_Seq\_Slope}: anclaje biométrico estable.}
La pendiente de la Secuencia Principal es la métrica más robusta de
todo el análisis: $\Delta\% = -0.35\%$, $d = 0.013$ (efecto
negligible) y $p = 0.289$ (no significativo). Este resultado confirma
lo documentado por Bahill et al.\ \cite{bahill1975}: la eficiencia
neuromuscular del sistema sacádico es una característica altamente
estable incluso bajo fatiga de corta duración, lo que la convierte
en un ancla biométrica confiable en entornos de iluminación variable.

\subsection{Síntesis: Dominancia Unidimensional y Fragilidad del Sistema}

Los resultados permiten caracterizar con precisión el mecanismo de fallo
del sistema mediante el siguiente modelo conceptual:

\begin{equation}
	\vec{\mu}_{P11,tarde} = \vec{\mu}_{P11,man}
	+ \underbrace{\lambda_{Pupil} \cdot
		(\bar{X}_{Pupil,P5} - \bar{X}_{Pupil,man}) \cdot \hat{e}_{Pupil}}_{
		\text{convergencia en Pupil\_Mean } (\lambda=0.789)}
	+ \underbrace{\sum_{j \neq Pupil} \delta_j \hat{e}_j}_{\substack{
			\text{divergencia en} \\ \text{métricas cinemáticas}}}
	\label{eq:modelo_deriva}
\end{equation}

donde $\hat{e}_j$ es el vector unitario en la dirección de la característica
$j$ en el espacio de decisión. La ecuación expresa que el vector de
características de P11-tarde es el de P11-mañana desplazado hacia P5
\textit{únicamente} en la dimensión pupilar, mientras que las métricas
cinemáticas añaden perturbaciones en direcciones opuestas. El clasificador
colapsa hacia P5 porque el peso de Gini de \texttt{Pupil\_Mean} supera
a la suma de las contribuciones divergentes de las demás métricas.

\noindent\textbf{Conclusión e implicaciones de diseño:}
La confusión identificada es consecuencia de un fallo de \textbf{dominancia
	de característica}: un único descriptor morfológico (\texttt{Pupil\_Mean})
concentra tanto poder discriminativo que su deriva ambiental domina la
decisión del ensemble sobre las señales contrarias de siete métricas
cinemáticas. Esto señala dos líneas de mejora concretas para el sistema:
\textit{(i)} normalización del diámetro pupilar por iluminancia ambiental
estimada, de modo que \texttt{Pupil\_Mean} represente la anatomía
individual y no las condiciones de luz \cite{beatty1982, duchowski2017};
\textit{(ii)} re-ponderación del peso relativo de los descriptores
morfológicos frente a los cinemáticos en el ensemble, potenciando
\texttt{Main\_Seq\_Slope} (La pendiente de la secuencia principal) ---que demuestra ser el indicador más estable
($d = 0.013$, $p = 0.289$)--- como ancla biométrica principal
\cite{bahill1975}.

\section{Discusión General}

Los resultados presentados en este capítulo demuestran la viabilidad técnica y científica de la propuesta, validando las tres hipótesis fundamentales de la investigación:

\begin{enumerate}
	\item \textbf{Calidad de la Señal:} La implementación de redes neuronales (YOLOv8) sobre video de bajo costo permitió extraer señales oculométricas de alta fidelidad. La reconstrucción exitosa de la \textit{Main Sequence} ($R^2 > 0.90$) confirma que el sistema captura la dinámica fisiológica real del ojo y no ruido aleatorio.
	\item \textbf{Identificación Biométrica:} Se demostró que la forma de mirar es única. Mediante el análisis de características híbridas (anatomía pupilar + dinámica sacádica) y algoritmos de \textit{Random Forest}, se alcanzó una exactitud de clasificación superior al 83\%, validando el potencial del movimiento ocular como huella biométrica robusta ante suplantaciones.
	\item \textbf{Doble Propósito (Cursor + Seguridad):} La prueba de homografía confirmó que el mismo sensor utilizado para identificar al usuario puede servir simultáneamente como dispositivo de entrada. Aunque la experiencia de uso actual está condicionada a una calibración rigurosa, se establece el precedente para interfaces donde la autenticación es continua e invisible: el sistema verifica la identidad del usuario constantemente mientras este interactúa con el computador mediante su mirada.
\end{enumerate}