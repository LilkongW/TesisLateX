% --- CAPÍTULO I ---
\chapter{Planteamiento del Problema}

\section{El problema}

Esta investigación se fundamenta en la premisa de que el ojo proporciona información biológica y cinemática única. Morfológicamente, el patrón del iris y las dimensiones de la pupila actúan como identificadores biométricos robustos e irrepetibles. Dinámicamente, el análisis de los movimientos oculares, tales como vibraciones o anomalías en la trayectoria, permite inferir datos significativos sobre el estado fisiológico y neurológico del individuo.

Un aspecto tecnológicamente relevante de este campo es la capacidad de traducir la mirada en comandos de control para la interacción humano-máquina. Estas aplicaciones son críticas en medicina, ofreciendo soluciones de asistencia para personas con movilidad reducida, y en áreas como la neurociencia, interesadas en modelar el comportamiento humano.

El objetivo central es implementar un sistema integral de captura y procesamiento de patrones oculares. El sistema permitirá validar la identificación biométrica y abrir la posibilidad a la creación de herramientas de control por mirada que mejoren la calidad de vida de usuarios con discapacidades motoras.

Díaz y Rojas \cite{diaz2021} sostienen que:

\begin{quote}
	El sentido más importante que proporciona la mayor cantidad de información, es el sentido de la vista y por esto, el estudio de la actividad visual mediante la técnica del seguimiento ocular (eye-tracking), se ha convertido en un tema de interés durante los últimos años, para las investigaciones en educación matemática, que incorporan los avances de la neurociencia cognitiva (p. 39).
\end{quote}

La cita hace alusión especial a la educación en matemática, ya que las investigadoras utilizan el seguimiento ocular para comprender cómo los estudiantes procesan la lectura de problemas. El seguimiento ocular está siendo estudiado en todas las áreas del conocimiento.

La física proporciona herramientas para abordar y resolver problemas prácticos y actuales. Su capacidad para modelar, analizar y comprender fenómenos naturales permite el desarrollo de soluciones que trascienden teorías abstractas y encuentran aplicaciones directas para mejorar la vida de las personas.

Se plantean cuatro interrogantes que guiarán esta investigación:

\begin{enumerate}
	\item ¿Cómo implementar un sistema de captura de imágenes o videos del ojo?
	\item ¿De qué manera se pueden analizar los datos recolectados mediante técnicas de procesamiento de imágenes?
	\item ¿Cómo identificar patrones de movimiento ocular a partir de los datos obtenidos?
	\item ¿Cómo se puede evaluar el sistema para su aplicación en el control de un cursor?
\end{enumerate}

\section{Objetivos}

Esta investigación se estructura mediante objetivos específicos. Como indican Ramírez et al. \cite{ramirez2004}: ``los objetivos nos permiten dejar en claro cuál va a ser el alcance de nuestro trabajo de investigación, nos indican el punto de llegada, lo que queremos lograr'' (p. 56).

Se establece un objetivo general y cuatro objetivos específicos.

\subsection{Objetivo general}
Implementar un sistema de seguimiento ocular a partir de técnicas de procesamiento de imágenes para el análisis de patrones oculares.

\subsection{Objetivos específicos}
\begin{enumerate}
	\item Implementar un sistema de captura de imágenes o videos del ojo.
	\item Analizar los datos recolectados mediante técnicas de procesamiento de imágenes.
	\item Identificar patrones de movimiento ocular a partir de los datos obtenidos.
	\item Evaluar el sistema para su posible aplicación en el control de un cursor.
\end{enumerate}

\section{Justificación}

Esta investigación integra conocimientos de física y herramientas tecnológicas para abordar problemas reales en el campo de la asistencia médica y la interacción humano-máquina. Los principios de la física, especialmente en áreas como la óptica y el análisis computacional, se vinculan con la medicina para desarrollar alternativas que mejoren la calidad de vida.

El seguimiento ocular permite la comunicación sin necesidad de movimientos físicos o habla, lo que resulta especialmente útil para personas con discapacidades motoras derivadas de lesiones cerebrales, parálisis cerebral o accidentes cerebrovasculares.

La identificación de patrones de movimiento ocular permite desarrollar herramientas asistivas y aporta al avance interdisciplinario de la física aplicada, la tecnología y la medicina.

Como señala Arias \cite{arias2010}, en esta sección de la justificación ``deben señalarse las razones por las cuales se realiza la investigación y sus posibles aportes desde el punto de vista teórico o práctico'' (p. 105).

El análisis de imágenes permite observar fenómenos invisibles a simple vista, impulsando avances tecnológicos y científicos que mejoran la calidad de vida.

\section{Organización de la tesis}

Este trabajo se encuentra estructurado en cinco capítulos:

\begin{description}
	\item[Capítulo 1: El problema a estudiar.] Se plantea el problema de investigación, se establecen los objetivos y se justifica la importancia del estudio.
	
	\item[Capítulo 2: Marco teórico.] Se expone una revisión conceptual sobre los temas fundamentales relacionados con el procesamiento de imágenes, la extracción de información a partir de los datos y conceptos fisiológicos relevantes para la investigación, que sirve como sustento para el desarrollo del proyecto.
	
	\item[Capítulo 3: Metodología.] Se describe el enfoque metodológico adoptado, las etapas principales del proceso de recolección y análisis de datos, así como las herramientas y técnicas empleadas para el tratamiento de la información y el desarrollo experimental.
	
	\item[Capítulo 4: Resultados.] Se presentan los resultados obtenidos a partir del análisis de los datos, acompañados de su respectiva interpretación y discusión preliminar.
	
	\item[Capítulo 5: Discusión y conclusiones.] Se realiza una discusión detallada de los resultados en comparación con investigaciones previas, se exponen las conclusiones alcanzadas y se proponen recomendaciones para futuras investigaciones.
\end{description}

Finalmente, se incluyen varios anexos que complementan el trabajo, tales como programas utilizados, abreviaturas, glosario de términos y documentación técnica relevante.